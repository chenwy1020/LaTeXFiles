%!TEX program=xelatex
\documentclass[UTF8]{ctexart}

%%%%%% 导入包 %%%%%%
\usepackage{graphicx}
\usepackage{xcolor}
\usepackage{algorithm,algorithmicx}
\usepackage{algpseudocode}
\usepackage{amsmath,amssymb,amsthm}


%%%% 正文开始 %%%%
\begin{document}

\title{数值分析第九次作业}
%
\author{10200115 陈文宇}

\date{\today}

\maketitle{}

1.已知$f(x)=a \cos(nx) + b \sin(nx),a^{2}+b^{2} \neq 0 $,
  求$f(x)$的阶数不超过 $n-1$ 的最佳一致逼近三角多项式$T^{*}_{n-1}(x)$和$E^{*}_{n-1}(x)$。
  
  根据三角函数公式
  \begin{align*}
  	f(x)&= \sqrt{a^{2}+b^{2}} \, \cos(nx-\theta)\\
  		&= \sqrt{a^{2}+b^{2}} \, \cos n(x-\dfrac{\theta}{n})\\
  		&\leq \sqrt{a^{2}+b^{2}}
  \end{align*}
  其中$\theta = \arctan(\frac{b}{a}) \in (-\frac{\pi}{2},\frac{\pi}{2})$
  
  注意到$f(x)$在$[0,2\pi]$上有$2n+1$个交错取最大值和最小值的点,则
  $$f(x_{k})=(-1)^{k}\sqrt{a^{2}+b^{2}}$$
  其中
  \[
  \left\{
  \begin{aligned}
  	x_{k} = \dfrac{\theta}{n} + \dfrac{k\pi}{n},&\qquad\theta \geq 0,k=0,1,2\dots 2n-1\\
  	x_{k} = \dfrac{\theta}{n} + \dfrac{k\pi}{n},&\qquad\theta < 0,k=1,2,3\dots 2n
  \end{aligned}
  \right.
  \]
  显然的是若$T^{*}_{n-1}(x)=0$,则$f(x)-T^{*}_{n-1}(x)$有$2n$个交错点构成的交错点组,有定理5.1知,
  $T^{*}_{n-1}(x)=0$是其最佳一致逼近三角多项式\newpage

2.$f(x) = x^{3}$,求 \,$p^{*}_{2}(x) \in P_{2}(x)$,$s.t.$\\
\begin{center}
   $||f(x)-p^{*}_{2}(x)||_{\infty} = \inf ||f(x)-p_{2}(x)||_{\infty} ,x\in[-1,1]$ 
\end{center}
解:\quad 由定理知 $$p_{2}(x)=f(x)-\dfrac{1}{4}T_{3}(x)$$

将$T_{3}(x)=4x_{3}-3x$代入上式得$$p^{*}_{2}(x)=\frac{3}{4}(x)$$


\end{document}
