%!TEX program=xelatex
\documentclass[UTF8]{ctexart}

%%%%%% 导入包 %%%%%%
\usepackage{graphicx}
\usepackage{xcolor}
\usepackage{tabularx}
\usepackage{algorithm,algorithmicx}
\usepackage{algpseudocode}
\usepackage{amsmath,amssymb,amsthm}

%%%%%% 算法部分改为中文显示 %%%%%%%%%
\floatname{algorithm}{算法}
\renewcommand{\algorithmicrequire}{\textbf{输入:}}
\renewcommand{\algorithmicensure}{\textbf{输出:}}

%%%% 正文开始 %%%%
\begin{document}

\title{关于与刘小端在数值分析期末作业的代码部分雷同的分析报告}
%
\author{陈文宇}

\date{\today}

\maketitle{}
\begin{abstract}
	“主观交流和客观相似” 同“抄袭”之间的界限到底在何方
\end{abstract}

\newpage
\tableofcontents
\newpage

\section{关于问题的重述与思考}
第一问和第二问基本源于曾经的课堂问题,由于我和小端做完作业都会共享代码,所以这方面几乎一模一样就可以解释了。

同时我们的作业报告完全不同,这也充分说明我们并不构成抄袭关系。

关于第三问问题重述:
对于指定基函数和积分法,确定$f(x)=e^{x^{2}+x}$的最佳平方逼近函数。

思路来源于教材 第102页,定理1.1。主要是用Gauss公式来刻画Gram矩阵和右端向量,形成$Aa=b$
即可求解。一般的我们都希望的是可以借用从前的代码,而简单地获取到想要的结果,这就使得我们通常会借鉴我们共享的代码。




\section{代码结果的比较}

\section{论述}
  
\section{结语}

\end{document}
