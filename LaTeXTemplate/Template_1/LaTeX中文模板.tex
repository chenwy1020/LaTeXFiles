%!TEX program=xelatex
\documentclass[UTF8]{ctexart}

%%%%%% 导入包 %%%%%%
\usepackage{graphicx}
\usepackage{xcolor}
\usepackage{algorithm,algorithmicx}
\usepackage{algpseudocode}
\usepackage{amsmath,amssymb,amsthm}

%%%%%% 算法部分改为中文显示 %%%%%%%%%
\floatname{algorithm}{算法}
\renewcommand{\algorithmicrequire}{\textbf{输入:}}
\renewcommand{\algorithmicensure}{\textbf{输出:}}

%%%% 正文开始 %%%%
\begin{document}

\title{数值分析实习II课程}
%
\author{翟起龙}

\date{\today}

\maketitle{}

考虑x=(0,1),令$A=\{\frac{1}{k}\}_{k\in N^{+}}$ 显然,对于任意$\varepsilon>0$,令 $n=1+[\frac{1}{\varepsilon}]$,
则集合$\{\frac{1}{n},\frac{2}{n}, \dots,\frac{n-1}{n}\}$为A的有限$\varepsilon -$网,
但$x_{n}=\frac{1}{n}\rightarrow 0 \notin X$, X非列紧。


\section{模板示例}

LaTeX可以很方便地在pdf文件中输入数学公式并进行排版. 这是一个tex文件的例子.

\subsection{数列极限的定义}

对任意$\varepsilon>0$, 存在$\delta>0$, 使得当$|x-x_0|<\varepsilon$时, 有
\[
|f(x)-f(x_0)|<\varepsilon.
\]

\subsection{多行公式}
下面是多行公式以及公式对齐的例子. 更多特殊符号以及公式环境网上均有详细的讲解和例子.
\begin{align}
	-\Delta u = f,  &\quad\text{in } \Omega, \label{first_eq}
	\\
	u = 0, \quad &\text{on } \partial\Omega. \label{second_eq}
\end{align}
方程组, 矩阵也可以类似地写出来.
\[
\left\{
\begin{aligned}
	a + b = c
	\\
	d + e = f
\end{aligned}
\right.
\hspace{1cm}
\begin{pmatrix}
	1 & 2
	\\
	3 & 4
\end{pmatrix}
\]
也可以很方便地实现公式引用\eqref{second_eq}.

\subsection{算法环境}
利用algorithm宏包可以写算法.
\begin{algorithm}[h]
	\begin{algorithmic}[1]
		\Require 长度$a$, 宽度$b$, 高度$c$
		\Ensure 质量$d$, 体积$e$, 密度$f$
		\State 首先判断$b$和$e$的大小
		\State 然后进行加法运算
		\Function {Add}{$a$, $b$, $c$}
		\State $d=a+1$
		\If{$b<e$}
		\State $f=c+2$
		\EndIf
		\EndFunction
	\end{algorithmic}
	\caption{算法示例}
\end{algorithm}

\section{绘制图片表格}
一些LaTeX常用的功能.

\subsection{插入表格}
通常用于列出误差等数据.

\begin{table}[h]
	\centering
	\begin{tabular}{|l|l|l|}
		\hline
		误差 &  第一次 & 第二次 \\
		\hline
		0.1  &  0.2   & 0.3   \\
		\hline
		1.5e-2 & 3.3e-10 & 1e8 \\
		\hline
	\end{tabular}
\caption{表格示例}
\end{table}

\subsection{插入图片}
通常用于画函数图像.

\begin{figure}[h]
	\centering
	\includegraphics[width=5cm]{Pics/jlu.jpg}
	\caption{吉林大学校徽}
\end{figure}

\section{设置参考文献}
如果需要参考文献的话可以这样引用 \cite{Yan2004}, 中英文均可 \cite{Ma2022}.

\begin{thebibliography}{99}
	
	\bibitem{Yan2004}{严子谦, 尹景学, 张然. 数学分析. 北京: 高等教育出版社, 2004, 1-372.}
	
	\bibitem{Ma2022}{C. Ma, Q. Zhang, W. Zheng. A Fourth-Order Unfitted Characteristic Finite Element Method. SIAM J. Numer. Anal., 2022, 60(4), 2203-2224.}
	
\end{thebibliography}

\end{document}
